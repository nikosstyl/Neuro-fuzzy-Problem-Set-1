% !TeX spellcheck = en_US
\section{Problem 7}
\subsection{Question a}
Let a generic activation function be
\begin{equation}
S(x) = \left\{
\begin{array}{cc}
	x^k, & x>L\\[1mm]
	x^k \dfrac{\left(L+x\right)^m}{\left(L+x\right)^m + \left(L-x\right)^m}, & \left|x\right| \le L\\[1mm]
	0, & x < -L
\end{array}
\right.
\end{equation}
This function has some parameters $k,L,m$ called hyperparameters. By using them, this function can approximate others.

\subsubsection{Swish}
%Swish: $k=1, \ L=2e, \ m=e$\\
In order for our activation function to approximate Swish, we have to see the latter first.
Swish's function is 
\[
S(x) = \dfrac{x}{1+e^{-x}} = x\dfrac{e^x}{1+e^x}
\]
and is plotted in figure~\ref{fig:prob7_swish}. In order to determine the parameters, we have to take a look in the plotted function. When $x>0$, Swish seems to be almost $y=x$, thus $k=1$. If we look at $x<0$, we can se that Swish curves until it settles back to 0.
That point $x_1$ where $Swish (x_1) = 0$ is parameter $L$. After trying multiple values, we opted with $2e$ as this gives the smaller overall error in that region.
After filling $k,L$, only parameter $m$ is left to be found. In general, $x^e$ is a good approximation of $e^x$, but an approximation only. Just trying $m=e$ and plotting the created function gets us the plot in figure~\ref{fig:prob7_swish} which is really close to Swish.
Finally, the activation function can be written 
\[
S(x) = \left\{
\begin{array}{cc}
	x, & x>2e\\[1mm]
	x \dfrac{\left(2e+x\right)^e}{\left(2e+x\right)^e + \left(2e-x\right)^e}, & \left|x\right| \le 2e\\[1mm]
	0, & x < -2e
\end{array}
\right.
\]

\begin{figure}[H]
	\centering
	\includegraphics[width=.5\textwidth]{../Problem 7/prob7_swish.pdf}
	\caption{Swish activation function and its approximation.}
	\label{fig:prob7_swish}
\end{figure}

\subsubsection{Sigmoid}
Sigmoid's function is 
\[
S(x) = \dfrac{1}{1+e^{-x}} = \dfrac{e^x}{1+e^x}
\]
Just from this type, we understand that is $\frac{Swish(x)}{x}$. So, selecting parameters is easier this time as $m$ and $L$ will not change. Because of the previous equation, parameter $k$ must be decreased by $1$, thus $k=0$. Figure~\ref{fig:prob7_sigmoid} shows both the sigmoid and its approximation.

\begin{figure}[H]
	\centering
	\includegraphics[width=.5\textwidth]{../Problem 7/prob7_sigmoid.pdf}
	\caption{Sigmoid activation function and its approximation.}
	\label{fig:prob7_sigmoid}
\end{figure}


\subsubsection{ReLU}
ReLU is defined as 
\[
ReLU(x) = \max(0,x) = \left\{
\begin{array}{cc}
	0, & x<0\\
	x, & x \ge 0
\end{array}
\right.
\]
From this, we can already determine parameter $k$ to be $k=1$, as we want $x^k = x \Rightarrow k=1$.
Setting $L=0$ ensures that the transition point between the linear and non-linear parts of the function occurs at $x=0$, which is consistent with the behavior of ReLU.
Also, setting $m=0$ means that the denominator term becomes $1$ for both the linear and non-linear parts of the function, simplifying the expression.
The plotted functions are presented in figure~\ref{fig:prob7_relu}\\
With parameters those mentioned above, $S(x)$ is the following
\[
S(x) = \left\{
\begin{array}{cc}
	x, & x\ge 0\\
	0, & x < 0
\end{array}
\right.
\]

\begin{figure}[H]
	\centering
	\includegraphics[width=0.5\textwidth]{../Problem 7/prob7_relu.pdf}
	\caption{ReLU activation function and its approximation.}
	\label{fig:prob7_relu}
\end{figure}

\subsubsection{ELU}
ELU's equation is 
\[
S(x) = \left\{
\begin{array}{cc}
	x, & x \ge 0\\
	\alpha \left(e^x - 1\right), & x<0\\
\end{array}
\right., \quad \text{with} \ \alpha=\left(0.1...0.3\right)
\]
From the equation above we understand that this activation function is almost the same as ReLU, with only difference being treatment of negative numbers. So, we can choose the same parameters from ReLU and still be a fairly good approximation to ELU, thus $k=1,\ L=0,\ m=0$.
Both ELU and it's approximation are presented in figure~\ref{fig:prob7_elu}.

\begin{figure}[H]
	\centering
	\includegraphics[width=0.5\textwidth]{../Problem 7/prob7_elu.pdf}
	\caption{ELU activation function and its approximation}
	\label{fig:prob7_elu}
\end{figure}


\subsection{Question b}
In order to make our calculations a bit easier, we set $g(x) = L+x$ and $i(x) = L-x$, with $\dfrac{dg(x)}{dx} = 1$, $\dfrac{di(x)}{dx} = -1$, $\dfrac{dg(x)}{dL} = 1$ and $\dfrac{di(x)}{dL} = 1$.

\subsubsection{Derivative with respect to x}
In $x \in \left[-L, L\right]$, the derivative is calculated as
\[
\begin{gathered}
\dfrac{dS(x)}{dx} = kx^{k-1} \dfrac{g^m(x)}{\left(g(x)+i(x)\right)^m} + \\[1mm]
+ x^k\dfrac{mg^{m-1}(x) \left( g^m(x) + i^m(x) \right) - g^m(x) \left( mg^{m-1}(x) - mi^{m-1}(x)\right)}{\left( g^m(x) + i^m(x) \right)^2} = \\[1mm]
= kx^{k-1} \dfrac{g^m(x)}{\left(g(x)+i(x)\right)^m} + x^k \dfrac{mg^m(x) i^m(x) \left( \frac{1}{g(x)} + \frac{1}{i(x)} \right)}{\left(g^m(x) + i(x)\right)^2}
\end{gathered}
\]

So, activation's function derivative is 
\begin{equation}
\frac{dS(x)}{dx} = \left\{
\begin{array}{cc}
	kx^{k-1} \ , & x>L\\[1mm]
	kx^{k-1} \dfrac{g^m(x)}{\left(g(x)+i(x)\right)^m} + x^k \dfrac{mg^m(x) i^m(x) \left( \frac{1}{g(x)} + \frac{1}{i(x)} \right)}{\left(g^m(x) + i(x)\right)^2}\ , & \left|x\right| \le L\\[1mm]
	0\ , & x<L
\end{array}
\right.
\end{equation}

\subsubsection{Derivative with respect to k}
We know from algebra that $\dfrac{d}{dx}\left(a^x\right) = a^x \ln a$. So
\[
\dfrac{dS}{dk} = x^k \ln (k) \dfrac{\left(L+x\right)^m}{\left(L+x\right)^m + \left(L-x\right)^m}, \quad \text{for } x \in \left[-L...L\right]
\]
Thus, the derivative is
\begin{equation}
\frac{dS(x)}{dk} = \left\{
\begin{array}{cc}
	x^k \ln(k) \ , & x>L\\[1mm]
 	x^k \ln (k) \dfrac{g^m(x)}{g^m(x) + i^m(x)}\ , & \left|x\right| \le L\\[1mm]
	0\ , & x<L
\end{array}
\right.
\end{equation}

\subsubsection{Derivative with respect to L}
This derivative is similar to that in respect to $x$, so calculating it gives us
\[
\dfrac{dS}{dL} = x^k \dfrac{mg^{m-1}(x) \left( g^m(x) + i^m(x) \right) - g^m(x) \left( mg^{m-1}(x) + mi^{m-1}(x)\right)}{\left( g^m(x) + i^m(x) \right)^2} = 0 
\]
So, 
\begin{equation}
	\dfrac{dS(x)}{dL} = 0, \ \text{for all different L}
\end{equation}

\subsubsection{Derivative with respect to m}
\[
\begin{gathered}
\dfrac{dS}{dm} = x^k\dfrac{g^{2m}(x) \ln(g(x)) +g^m(x) i^m(x) \ln (g(x)) - \left( g^{2m}(x) \ln(g(x)) + g^m(x) i^m(x) \ln(i(x)) \right)}{\left( g^m(x) + i^m(x)\right)^2} = \\[1mm]
= \dfrac{g^m(x) i^m(x) \ln(g(x)) - g^m(x) i^m(x) \ln(i(x))}{\left(  g^m(x) + i^m(x)\right)^2} 
= \dfrac{g^m(x) i^m(x) \ln\left(\frac{g(x)}{i(x)}\right)}{\left(  g^m(x) + i^m(x)\right)^2}
\end{gathered}
\]

Thus, the derivative of $S(x)$ with respect to m is 
\begin{equation}
\dfrac{dS}{dm} = \left\{
\begin{array}{cc}
	0 \ , & x>L\\[1mm]
	\dfrac{g^m(x) i^m(x) \ln\left(\frac{g(x)}{i(x)}\right)}{\left(  g^m(x) + i^m(x)\right)^2}\ , & \left|x\right| \le L\\[1mm]
	0\ , & x<L
\end{array}
\right.
\end{equation}

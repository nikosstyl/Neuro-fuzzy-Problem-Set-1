% !TeX spellcheck = en_US
\section{Problem 11}

Fuzzy logic is a type of logic that deals with vague, imprecise, or uncertain information. It is based on the concept of fuzzy sets, which are sets that can have any degree of membership between $0$ and $1$.  The value $0$ is used to represent complete non-membership, the value $1$ is used to represent complete membership, and values in between are used to represent intermediate degrees of membership.This means that an element can be a member of a fuzzy set to some degree, rather than all or nothing.

The uniqueness of fuzzy logic is that fuzzy logic can handle imprecise and uncertain information,which makes it a valuable tool for dealing with real-life problems that are inherently vague or fuzzy. 

On this exercise, we are dealing with the linguistic variable \textit{Truth} with a possible membership set: 

\begin{center}
	\textit {T = \{Absolutely false, Very false, False, Fairly true, True, Very true, Absolutely true}\}
\end{center}

Based on that set we may define the membership function of truth as:
\begin{center}
	\textit{True(u) = u \hspace{3mm}  False(u) = 1-u}
\end{center}
for each $u \in [0, 1]$.

\subsection{Question a}
We have the following statement: \textit{If the statement "A is true" has truth value equal to x, then the statement "A is very true" has truth value $x^2$, because the "very true" is more demanding.\\
\textbf{Q: }Let "S" be a fuzzy set. Then "Very S" is a fuzzy subset of "S".} True or False?\\
\textbf{A: }False. The statement "Very S is a fuzzy subset of S" is not necessarily true. In fuzzy logic, the term "very" is a linguistic modifier that acts as an intensifier. If it is applied to a fuzzy set, it does not create a subset, but changes the degrees of membership of the elements in the set. In other words, the word "very" expresses the strength of the membership function T.
\subsection{Question b}
"More or less" is used as an adjective to increase vagueness - the interpretation is that if "A is true" has truth value x, then "A is more or less true" has truth value $\sqrt{x}$.\\
\textbf{Q: }Let "S" be a fuzzy set. Then "S" is a fuzzy subset of "more or less S".\\
\textbf{A: }False. Just like with the statement "Very true" the term "more or less" is a linguistic modifier to increase vagueness. It also expresses the strength of the membership function. To justify our explanation we will cite an example:\\
If "The students are happy" has a truth value of $0.25$, then "The students are more or less happy" would have a truth value of $\sqrt(0.25)$, which equals $0.5$.\\
However, this interpretation assumes that the "more or less" modifier always applies the square root function to the truth value, which may not always be the case in every context or with every fuzzy logic system. The actual effect of the "more or less" modifier can depend on the specific definitions and rules of the fuzzy logic system being used.
\subsection{Question c}
\textbf{Q: }Using the definitions just given, is it true that "not very S" is a subset of "more or less S", or vice versa, or is it impossible to say?\\
\textbf{A: }It is impossible to make a definitive statement without additional information about the specific interpretations of "not very S" and "more or less S" in the context of the fuzzy set theory you are using.\\
The above definitions state that "very" is used to decrease the vagueness of belonging to a fuzzy set, and that "more or less" is used to increase vagueness. However, the specific interpretations of "not very S" and "more or less S" depend on the underlying logic or semantics of the fuzzy set theory you are using.\\
Without a clear understanding of how "not very" and "more or less" are defined in your particular fuzzy set framework, it is generally not possible to conclusively determine whether "not very S" is a subset of "more or less S" or vice versa. This would require a precise definition and understanding of the relationships between these modifiers in the given context.
\subsection{Question d}
\textbf{Q: }Is "not more or less S" a subset of "very S", or vice versa, or is it impossible to say?\\
\textbf{A: } It's impossible to say. The terms "not", "more or less", and "very" are linguistic modifiers in fuzzy logic that modify the membership grades of elements in a fuzzy set, and their effects can depend on the specific system.In general, "not S" would be a new fuzzy set where the membership grade of each element is the complement of its grade in "S". "More or less S" would increase the vagueness or uncertainty of the membership grades, and "very S" would intensify the membership grades.\\
Without knowing the specific membership grades of the elements in "S" and how the "not", "more or less", and "very" modifiers are defined in the system, it's impossible to definitively say whether "not more or less S" is a subset of "very S", or vice versa.



	






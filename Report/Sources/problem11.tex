% !TeX spellcheck = en_US
\section{Problem 11}

Fuzzy logic is a type of logic that deals with vague, imprecise, or uncertain information. It is based on the concept of fuzzy sets, which are sets that can have any degree of membership between 0 and 1.  The value zero is used to represent complete non-membership, the value one is used to represent complete membership, and values in between are used to represent intermediate degrees of membership.This means that an element can be a member of a fuzzy set to some degree, rather than all or nothing.

The uniqueness of fuzzy logic is that fuzzy logic can handle imprecise and uncertain information,which makes it a valuable tool for dealing with real-life problems that are inherently vague or fuzzy. 

On this exercise, we are dealing with the linguistic variable \textit{Truth} with a possible membership set: 

\begin{center}
	\textit{T = {Absolutely false, Very false, False, Fairly true, True, Very true, Absolutely true}}
\end{center}

Based on that set we may define the membership function of truth as:
\begin{center}
	\textit{True(u) = u \hspace{3mm}  False(u) = 1-u}
\end{center}
for each $u \in [0, 1]$.

	






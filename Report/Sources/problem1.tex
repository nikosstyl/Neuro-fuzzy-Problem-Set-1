\section{Πρόβλημα 1}

Τα contour lines της συνάρτησης $f(x,y)$ που δίνεται στην εκφώνηση παράγεται με τον ακόλουθο κώδικα MATLAB και πορουσιάζεται στην εικόνα~\ref{fig:prob_1_contour_lines}.

\begin{lstlisting}[]
function [Z] = plot_contour(start_num, end_num)
	
	x = linspace(start_num, end_num, 100);
	y = x;
	[X, Y] = meshgrid(x, y);
	Z = X.^2 + 4*X.*Y + Y.^2;
	contour(X, Y, Z, 40);
	xlabel('X');
	ylabel('Y');
end
\end{lstlisting}
\begin{figure}[h]
	\centering
	\includesvg[width=0.5\textwidth]{../Problem 1/contour_lines.svg}
	\caption{Οι contour lines της συνάρτησης $f(x,y)$}
	\label{fig:prob_1_contour_lines}
\end{figure}

%Η συνάρτηση $f$ είναι μορφής quadratic, αλλά υπό μορφή πινάκων γράφεται όπως παρακάτω:
%\begin{equation}
%\begin{gathered}
%f(x,y) = x^2 + 4xy + y^2 = x^2 + 2xy + 2xy + y^2 = \left( x^2 + 2xy \right) + \left( 2xy + y^2 \right), \quad \text{άρα} \\ 
%Q(x) = X^T A \MathSpace X = 
%\left[
%\begin{array}{cc}
%	x & y
%\end{array}
%\right]
%\left[
%\begin{array}{cc}
%	1 & 2 \\
%	2 & 1 \\
%\end{array}
%\right]
%\left[
%\begin{array}{c}
%	x\\y
%\end{array}
%\right]
%\end{gathered}
%\end{equation}

Η γενική εξίσωση μιας quadratic συνάρτησης είναι η $f(x,y) = ax^2 + 2bxy + cy^2$, παίρνοντας έτσι τους συντελεστές $a=1, \MathSpace b=2, \MathSpace c=1$.
Ο υπολογισμός της διακρίνουσας μπορεί να μας βοηθήσει να υπολογίσουμε το σημείο όπου βρίσκεται το τοπικό ακρότατο της συνάρτησης. 
\begin{equation}
\begin{gathered}
D =
\left[
\begin{array}{cc}
	f_{xx} & f_{xy} \\
	f_{yx} & f_{yy} \\
\end{array}
\right]
= f_{xx} f_{yy} - f^2_{xy} = 2 \times 2 - 4^2 = -12 < 0, \quad \text{όπου} \\
f_{xx} = \frac{\partial^2 f}{\partial x^2} = 2, \quad
f_{yy} = \frac{\partial^2 f}{\partial y^2} = 2, \quad
f_{xy} = \frac{\partial}{\partial y} \left( \frac{\partial f}{\partial x} \right) = 4
%D = 2 \times 2 - 4^2 = -12 < 0.
\end{gathered}
\end{equation}
´Άρα, αρκεί να βρούμε το σημείο όπου $\frac{\partial f}{\partial x}$ και $\frac{\partial f}{\partial y}$ ισούνται με 0. Έτσι, το σημείο αυτό θα είναι saddle point ή σημείο καμπής όπου οι κλίσεις στις ορθογώνιες κατευθύνσεις είναι όλες μηδέν, αλλά το οποίο δεν αποτελεί τοπικό άκρο της συνάρτησης.
Συγκεκριμένα έχουμε: 
\begin{equation}
\left\{
\begin{array}{c}
	\frac{\partial f}{\partial x} = 2x + 4y = 0 \\ 
	\frac{\partial f}{\partial y} = 4x + 2y = 0 \\
\end{array}
\right.
\Rightarrow
\left\{
\begin{array}{c}
	x = 0\\y=0\\
\end{array}
\right.
\end{equation}
Άρα, το σημείο $(x,y) = (0,0)$ είναι το saddle point για την συνάρτηση που δίνεται και αυτό γίνεται γνωστό και από τις contour lines της συνάρτησης.
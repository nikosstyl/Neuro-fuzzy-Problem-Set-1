% !TeX spellcheck = en_US

\section{Problem 2}
Gradient Descent is an optimization algorithm used to minimize a function by iteratively moving in the direction of steepest descent defined by the negative value of the gradient. It is mostly used to update the parameters of models.\\
In this exercise we need to execure two iterations of the Gradient Descent to the function
\begin{center}
$	f(x_1,x_2) = (x_1+2 \cdot x_2 - 7)^2 + (2 \cdot x_1 + x_2 -5)^2$
\end{center}
with initial point $x_0 = (-9.5,9.5)$.

The steps that we must use each time are specific.
\begin{center}
	\textit{FIRST ITERATION k = 0}
\end{center}

\underline{Step1: Calculate the Gradient}\\

\(\nabla f(x_1,x_2) = \left(\begin{array}{c}
	\frac{\partial f}{\partial x_1} \\[1mm]
	\frac{\partial f}{\partial x_2}
\end{array}\right)\) = $\left(\begin{array}{c}
 2 \cdot (x_1 + 2 \cdot x_2 - 7) + 4 \cdot (2 \cdot x_1 + x_2 -5)\\[1mm]
4 \cdot (x_1 + 2 \cdot x_2 -7) + 2 \cdot (2 \cdot x_1 + x_2 - 5)
\end{array}\right) = \left(\begin{array}{c}
10 \cdot x_1 + 8 \cdot x_2 - 34\\[1mm]
8 \cdot x_1 + 10 \cdot x_2 -38
\end{array}\right)$ \\[3mm]

where at the point $x_0 = (-9.5,9.5)$ we have $(\nabla f(x) = \left(\begin{array}{c}
	-53 \\
	-19
\end{array}\right)$
\\[4mm]

\underline{Step2: Calculate the norm \(\|f\|\).}\\

\(\|\nabla f(x_1,x_2)\| = \sqrt{\left(\frac{\partial f}{\partial x_1}\right)^2 + \left(\frac{\partial f}{\partial x_2}\right)^2} = \sqrt{\left(10 \cdot x_1 + 8 \cdot x_2 - 34\right)^2 + \left(8 \cdot x_1 + 10 \cdot x_2 -38\right)^2}\) \\[2mm]

we want the \(\|f(x_1,x_2)\|\) at the point $x_0 = (-9.5,9.5)$. so by substituting we have \(\|f(x_1,x_2)\| = 56.3028\).
\\[7mm]
\underline{Step3: Find the direction of descending at $x_0 = (-9.5,9.5)$}\\

$S_0 = -\dfrac{\nabla f(x_1,x_2)}{\|f\|} = -\dfrac{1}{56.3028} \cdot \left(\begin{array}{c}
	-53 \\
	-19
\end{array}\right)$ $\rightarrow S_0 = \left(\begin{array}{c}
0.9413 \\
0.3375
\end{array}\right)$
\\[7mm]
\underline{Step4: Find next step and optimal $\lambda$}\\[2mm]
Each step is characterized by $x_{k+1} = x_k + \lambda_k \cdot S_k$
\begin{equation}
	x_1 = x_0^T + \lambda_0 \cdot S_0 = \left(\begin{array}{c}
		-9.5 \\
		9.5
	\end{array}\right) + \lambda_0 \cdot  \left(\begin{array}{c}
		0.9413 \\
		0.3375
	\end{array}\right) \\
	 \rightarrow x_1 = \left(-9.5+0.9413 \cdot \lambda_0, 9.5+0.3375 \cdot \lambda_0\right)
	 \label{eq}
\end{equation}
\\[0.5mm]
Now we replace to the $f(x_1,x_2)$ function the $x_1$ from (\ref{eq}).
\begin{equation}
	\centering
	\begin{gathered}
		f(x_1) = \left(-9.5+0.9413\cdot\lambda_0+12+2 \cdot \lambda_0 \cdot 0.3375 \right)^2+\left(9.5+2 \cdot \lambda_0 \cdot 0.9413+0.3375 \cdot \lambda_0-24\right)^2
	\end{gathered}
\end{equation}
\label{eq4}

In order to find the optimal $\lambda$ we must follow some more steps.\\
We should set the derivative of $f(x_1)$ relative to $\lambda_0$ equal to zero.\\

\[
\begin{gathered}
	\dfrac{\partial f(x_1)}{\partial \lambda_0} = 2 \cdot (-9.5+0.9413 \cdot \lambda_0+12+2 \cdot \lambda_0\cdot 0.3375) \cdot (0.9413+2 \cdot 2 \cdot 0.3375)\\
	 +2 \cdot(9.5+2 \cdot \lambda_0 \cdot 0.9413+0.3375 \cdot \lambda_0 -24) \cdot (2 \cdot 0.9413+0.3375)\\
	 = 2 \cdot [1.6163\cdot(1.6163 \cdot \lambda_0+2.5)+2.2201\cdot(2.2201\cdot\lambda_0-14.5)]\\
	 \rightarrow \dfrac{\partial f(x_1)}{\partial \lambda_0} = 7.5412 \cdot \lambda_0-28.1506
\end{gathered}
\]

We set $\dfrac{\partial f(x_1)}{\partial \lambda_0} = 0$ and because $\lambda \geq 0$,\\

$7.5412 \cdot \lambda_0-28.1506 = 0 \rightarrow \fbox{{\(\lambda_0 = 3.7329\)}}$ \\

and from equation \ref{eq},by replacing the $\lambda_0$, we get $x_1 = (-5.9862,10.7599)$
\\[2mm]
\begin{center}
	\textit{SECOND ITERATION k = 1}
\end{center} 
We will follow all 4 steps again to complete the second iteration for $x_1 = (-5.9862,10.7599)$.\\

\underline{Step1- Gradient at $x_1$:}
$\nabla f(x_1) = \left(\begin{array}{c}
	-7.7828\\
	21.7094
\end{array}\right)$\\

\underline{Step2- Norm at $x_1$:}
$(\|\nabla f(x_1)\|) = 23.0623$\\

\underline{Step3- Direction of descending at $x_1$:}
 $S_1 = -\dfrac{\nabla f(x_1)}{\|f(x_1)\|} = -\dfrac{1}{23.0623} \cdot \left(\begin{array}{c}
	-7.7828\\
	21.7094
\end{array}\right)$ $\rightarrow S_1 = \left(\begin{array}{c}
	0.3375 \\
	-0.9413
\end{array}\right)$\\

\underline{Step4- Find next step and optimal $\lambda$}\\[2mm]
\begin{equation}
	\begin{gathered}
	x_2 = x_1^T + \lambda_1 \cdot S_1 = \left(\begin{array}{c}
	    -5.9862 \\
		10.7599
	\end{array}\right) + \lambda_1 \cdot  \left(\begin{array}{c}
		0.3375 \\
		-0.9413
	\end{array}\right) \\[1mm]
	\rightarrow x_2 = \left(-5.9862+0.3375 \cdot \lambda_1, 10.7599-0.9413 \cdot \lambda_1\right)		
\end{gathered}
\label{eq5}
\end{equation}
\\[0.5mm]

Again we replace to the $f(x_1,x_2)$ function the $x_2$ from (\ref{eq5}).
\begin{equation}
	\centering
	\begin{gathered}
		f(x_2) = \left(8.5336+0.3375 \cdot \lambda_1-2 \cdot \lambda_1 \cdot 0.9413 \right)^2+\left(-0.9413 \cdot \lambda_1+2 \cdot \lambda_1 \cdot 0.3375-6.2125\right)^2
	\end{gathered}
\end{equation}
\label{eq6}

Subsequently, we will find the derivative of $f(x_2)$\\
\[
\begin{gathered}
	\frac{\partial f(x_2)}{\partial \lambda_1} = 2\cdot[(8.5336+0.3375 \cdot \lambda_1-2 \cdot \lambda_1 \cdot 0.9413) \cdot (-1.5451)
	+ \\ \cdot(-0.9413 \cdot \lambda_1+2 \cdot \lambda_1 \cdot 0.3375-6.2125) \cdot (-0.2663)]\\
	= 2 \cdot [-1.5451 \cdot (-1.5451 \cdot \lambda_1+8.5336)+(-0.2663)\cdot(-0.2663\cdot\lambda_1-6.2125)] \\[2mm]
	\rightarrow \frac{\partial f(x_2)}{\partial \lambda_1} = 2\cdot[2.4582 \cdot \lambda_1-11.5309]
\end{gathered}
\]\\

We set $\dfrac{\partial f(x_2)}{\partial \lambda_1} = 0$ and because $\lambda \geq 0$,\\

$2\cdot[2.4582 \cdot \lambda_1-11.5309] = 0 \rightarrow \fbox{{\(\lambda_1 = 4.6908\)}}$ \\

and from equation \ref{eq5},by replacing the $\lambda_1$, we get $x_2 = (-4.4031,6.3444)$.
\\[4mm]
The results of the execution of the two iterations are shown in the following table.

\begin{table}[H]
	\centering
	\begin{tabular}{|c|c|c|}
		\hline
		%\centering
		\textbf{\# of iterations} & $\mathbf{x_k+1}$ & $\mathbf{\lambda_k}$ \\ \hline
		k = 0 & (-5.9862,10.7599) & 3.7329 \\ \hline
		k = 1 & (-4.4031,6.3444) & 4.6908\\
		\hline
	\end{tabular}
	\caption{Results}
\end{table}











